\documentclass[a4paper,10pt]{scrreprt}
\usepackage[utf8]{inputenc}
\usepackage[ngerman]{babel}
\usepackage{lmodern}
\usepackage[juratotoc]{scrjura}
\usepackage[]{scrpage2}
\usepackage{graphicx}
\usepackage[left=3cm, right=2.5cm, top=2.65cm, bottom=2.65cm]{geometry}

\newcommand{\shiftleft}[2]{\makebox[0pt][r]{\makebox[#1][l]{#2}}}

\rofoot {
	\raisebox{-2.7cm}{
		\shiftleft{0.88\paperwidth}{
			\includegraphics[width=\paperwidth]{img/footer-line.pdf}
		}
	}
}

\thispagestyle{scrheadings}
\pagestyle{scrheadings}

\sloppy

\useshorthands{'}
\defineshorthand{'S}{\Sentence\ignorespaces}

\setlength{\parindent}{0pt}
\setlength{\parskip}{10pt}

\usepackage{hyperref}
\hypersetup{
	pdfauthor={Kreativität trifft Technik e.V.},
	pdftitle={Hausordnung und Umgangsregeln}
}

\begin{document}

\title{Hausordnung und Umgangsregeln}
\subtitle{Vorschlag von sre (nicht beschlossen!)}
\author{Kreativität trifft Technik}

\makeatletter
\begin{titlepage}

\newcommand{\HRule}{\rule{\linewidth}{0.5mm}} % Defines a new command for the horizontal lines, change thickness here

\center % Center everything on the page
 
%----------------------------------------------------------------------------------------
%   HEADING SECTIONS
%----------------------------------------------------------------------------------------

\textsc{\LARGE \@author}\\[1.5cm] % Name of your university/college

%----------------------------------------------------------------------------------------
%   TITLE SECTION
%----------------------------------------------------------------------------------------

\HRule \\[0.4cm]
{ \huge \bfseries \@title}\\[0.4cm] % Title of your document
\HRule \\[1.5cm]
 
% If you don't want a supervisor, uncomment the two lines below and remove the section above
%\Large \emph{Author:}\\
%John \textsc{Smith}\\[3cm] % Your name

%----------------------------------------------------------------------------------------
%   DATE SECTION
%----------------------------------------------------------------------------------------

{\large \@subtitle}\\[2cm] % Date, change the \today to a set date if you want to be precise

%----------------------------------------------------------------------------------------
%   LOGO SECTION
%----------------------------------------------------------------------------------------

\includegraphics[scale=2.0]{img/logo.pdf}\\[1cm] % Include a department/university logo - this will require the graphicx package
 
%----------------------------------------------------------------------------------------

\vfill % Fill the rest of the page with whitespace

\end{titlepage}


\begin{contract}

\Paragraph{title={Allgemeines}}

'S Mit Betreten der Vereinsräume wird die Hausordnung des
Vereins \textit{Kreativität trifft Technik e.V.} akzeptiert.

'S Das Hausrecht wird von den Vereinsmitgliedern durchgesetzt.
'S Der Keyholder trägt die Verantwortung.

'S Die Räume sollten sich immer in einem repräsentativen Zustand befinden.

'S Mit Resource wie Strom, Wasser, Internet und Lagerplatz ist sparsam umzugehen.

\Paragraph{title={Keyholder}}

'S Der Keyholder ist das Mitglied, welcher den Space geöffnet hat. Sollte der
Keyholder die Räumlichkeiten verlassen, so überträgt er den Status vorher an
eine andere legitimierte Person und dokumentiert dieses in geeigneter Weise.

'S Der Vorstand legt fest, welche Mitglieder potentielle Keyholder sind.

'S Mitglieder können die Räume betreten, sobald ein Keyholder vor Ort ist
und den Space geöffnet hat.

'S Der Keyholder trägt die Verantwortung für die Räumlichkeiten.

\Paragraph{title={Gegenstände}}

'S Fremdes Eigentum/Werkzeug ist zu respektieren.

'S Jedes Mitglied und jeder Gast hat auf seine Sachen selbst zu achten.
'S Der Verein oder seine Mitglieder übernehmen keine Haftung für mitgebrachte
Gegenstände jeglicher Art.
'S Eine Ausnahme bilden Dauerleihgaben, welche vom Vorstand akzeptiert wurden.

'S Mitglieder dürfen mitgebrachte Gegenstände, falls nicht anders abgesprochen,
nur in ihrem Schließfach lagern.
'S Die Schließfächer dürfen nicht zum Lagern von verbotenen Stoffen verwendet
werden.
'S Die Schließfächer dürfen nicht für das Lagern von gefährlichen Stoffen und
Gütern (GSG), wie z.B. Säuren und Laugen, verwendet werden.

'S Für die Lagerung von Gegenständen außerhalb von Schließfächern gelten folgende
Regeln:
\begin{enumerate}
	\item Es ist eine Absprache mit dem Vorstand nötig.
	\item Temporär kann der Keyholder Platz zuweisen, bis eine Entscheidung
	vom Vorstand vorliegt.
	\item Die Gegenstände sind mit einem (ausgefüllten) Nutzer-Sticker versehen.
	\item Das Recht zur Lagerung kann jederzeit wieder entzogen werden; in diesem
	Fall sollten die Gegenstände so früh wie möglich wieder mitgenommen werden.
\end{enumerate}

'S Nicht beschriftete Gegenstände sollten dem Besitzer, oder, falls dieser
nicht bekannt ist, dem Keyholder gemeldet werden.
'S Wenn der Besitzer nicht ermittelt werden kann wird der Gegestand als
Vereinseigentum angesehen und entweder entsprechend beschriftet oder
weggeschmissen!

\Paragraph{title={Gegenseitige Rücksichtnahme}}

'S Es gelten allgemein von der Gesellschaft anerkannte Regeln und Normen.

'S Das Rauchen ist in den Räumlichkeiten des Vereins untersagt.

'S Bei Bild- und Tonaufnahmen sind das Persönlichkeitsrecht und das Recht der
Abgebildeten am eigenen Bild zu beachten.

'S Teamarbeit ist explizit erwünscht.
'S Um ein angenehmes Arbeiten für alle zu ermöglichen sollte versucht werden
unsere 1/3 Faustregel einzuhalten: 33\% helfen, 33\% geholfen werden, 33\%
selbstständiges arbeiten.

'S Die Lautstärke darf das Maß, welches alle Anwesenden akzeptieren, nicht
überschreiten.

\Paragraph{title={Ordnung}}

'S Nach Benutzung von Arbeitsplätzen und dem Loungebereich hat jeder seinen
Müll und Dreck zu entsorgen, aufzuräumen bzw. zu säubern.
'S Benutztes und herumliegendes Werkzeug ist wieder an den dafür vorgesehenen
Platz einzusortieren.
'S Der Arbeitsplatz muss für den Nächsten wieder benutzbar sein.

'S Bemühe dich um eine allgemeine Ordnung.
'S Fällt dir Müll oder Unordnung auf, die nicht zu dir gehört: räume dies auf.

'S Alle Wasserstellen (Waschbecken, Wasserhahn, Toilette, Dusche, ...) sind
nach Benutzung sauber zu hinterlassen.

'S Benutzes Geschirr ist vor Verlassen der Räumlichkeiten in den
Geschirrspüler zu räumen.
'S Sollte der Geschirrspüler voll sein, so ist dieser anzuschalten.
'S Sollte der Geschirrspüler fertig sein, so ist dieser auszuräumen.

'S Verschimmeltes und Gammeliges wird entfernt, egal wem es gehört.
'S Diesbezügliche Experimente sind beim Keyholder anzumelden, entsprechend zu
kennzeichnen, und dürfen die Gesundheit anderer Personen nicht gefährden.

\end{contract}
\end{document}
