\documentclass[a4paper,10pt]{scrreprt}
\usepackage[utf8]{inputenc}
\usepackage[ngerman]{babel}
\usepackage{lmodern}
\usepackage[juratotoc]{scrjura}
\usepackage[]{scrpage2}
\usepackage{graphicx}
\usepackage[left=3cm, right=2.5cm, top=2.65cm, bottom=2.65cm]{geometry}

\newcommand{\shiftleft}[2]{\makebox[0pt][r]{\makebox[#1][l]{#2}}}

\rofoot {
	\raisebox{-2.7cm}{
		\shiftleft{0.88\paperwidth}{
			\includegraphics[width=\paperwidth]{img/footer-line.pdf}
		}
	}
}

\thispagestyle{scrheadings}
\pagestyle{scrheadings}

\sloppy

\useshorthands{'}
\defineshorthand{'S}{\Sentence\ignorespaces}

\setlength{\parindent}{0pt}
\setlength{\parskip}{10pt}

\usepackage{hyperref}
\hypersetup{
	pdfauthor={Kreativität trifft Technik e.V.},
	pdftitle={Satzung des Vereins Kreativität trifft Technik}
}

\begin{document}

\title{Satzung des Vereins\\Kreativität trifft Technik}
\subtitle{
	Errichtet auf der Gründungsversammlung am 11. Juli 2011\\
	Zuletzt geändert auf der Mitgliederversammlung am 2. November 2013
}
\author{Kreativität trifft Technik}

\input{title.tex}

\section*{Präambel}

ENTSCHLOSSEN, den technischen Fortschritt sowie die digitale, kulturelle und
gesellschaftliche Entwicklung voranzutreiben,

IN DEM BESTREBEN, die Vermittlung von Wissen über Technik und deren
verantwortungsvollen, Nutzen bringenden und nachhaltigen Einsatz in der
vernetzten Informationsgesellschaft zu fördern,

MIT DEM WUNSCH, freie Kommunikation zu stärken, digitale Chancengleichheit
anzustreben und durch den Einsatz von Technologie das Leben von Menschen zu
verbessern,

IN DEM FESTEN WILLEN, Technologie und Wissen zum Wohle der individuellen und
gesellschaftlichen Entwicklung einzusetzen,

MIT DEM VORSATZ, die Verwendung freier und offener Lizenzen im
Immaterialgüterbereich weiter zu etablieren,

IN ACHTUNG religiöser wie auch politischer Neutralität und der Unabhängigkeit
von Interessen Dritter,

IN BEKENNTNIS zu Freiheit und Demokratie,

IN ANERKENNUNG der Grund- und Menschenrechte sowie der Rechtsstaatlichkeit,

IN BESTÄTIGUNG des humanistischen Menschenbildes und

UNABHÄNGIG von persönlichen materiellen Bereicherungsabsichten

sind wir wie folgt ÜBEREINGEKOMMEN:

\newpage

\begin{contract}
\Clause{title={Name, Sitz und Geschäftsjahr}}

'S Der Verein führt den Namen „Kreativität trifft Technik“.

'S Der Verein hat seinen Sitz in Oldenburg (Oldb.).

'S Er soll in das Vereinsregister eingetragen werden und den Zusatz „e.V.“
führen.

'S Das Geschäftsjahr ist das Kalenderjahr.

\Clause{title={Zweckbestimmung und Gemeinnützigkeit}}

'S Der Verein verfolgt ausschließlich und unmittelbar gemeinnützige Zwecke im
Sinne des Abschnitts „Steuerbegünstigte Zwecke“ der Abgabenordnung.

'S Zweck des Vereins ist die Förderung von Wissenschaft und Forschung, die
Förderung der Volksbildung sowie die Förderung von Kunst und Kultur.
'S Der Satzungszweck wird in Ansehung der Präambel insbesondere verwirklicht
durch

\begin{enumerate}
    \item den freien, interdisziplinären Austausch von Wissen auf dem Gebiet
	      der Informatik, der Technik nebst den Natur- und
		  Gesellschaftswissenschaften, auch durch die Herausgabe von
		  freizugänglichen Schriften und Büchern in elektronischer Form;
    \item die Organisation, Durchführung und Unterstützung von
	      Informationsveranstaltungen, Workshops, Seminaren und Konferenzen zur
          \begin{enumerate}
              \item Förderung der Technik- und Medienkompetenz von Jugendlichen
			        und Erwachsenen;
              \item Aufklärung über Risiken und Gefahren (digitaler) Technik,
			        Medien und Datennetze, aber gleichfalls auch zur Aufklärung
					über deren Potentiale für die gesellschaftliche, kulturelle,
                    demokratische, technologische und wirtschaftliche
					Entwicklung und die sich damit eröffnenden individuellen
					und zivilgesellschaftlichen Entfaltungsmöglichkeiten;
          \end{enumerate}
    \item die Förderung von digitaler sowie technischer Kunst und Kultur nebst
	      dem interdisziplinären Austausch darüber;
    \item die Förderung von schöpferisch-kritischem Umgang mit Technologie
	      sowie den interdisziplinären Austausch darüber;
	\item die Veranstaltung von Ausstellungen für kulturelle oder
	      Unterrichtszwecke;
	\item den Betrieb und die Unterhaltung eines Hackspaces;
	\item den Austausch mit anderen nationalen und internationalen
	      Gruppierungen, deren Zwecke und Ziele mit denen des Vereins
		  vereinbar sind.
\end{enumerate}

'S Der Verein ist selbstlos tätig und verfolgt nicht in erster Linie
eigenwirtschaftliche Ziele.
'S Mittel des Vereins dürfen nur für die satzungsmäßigen Zwecke verwendet
werden.
'S Die Mitglieder erhalten keine Zuwendungen aus den Mitteln des Vereins.
'S Keine Person darf durch unverhältnismäßig hohe Vergütungen oder durch
Ausgaben, die dem Zweck des Vereins fremd sind, begünstigt werden.

\Clause{title={Mitgliedschaft}}

'S Der Verein besteht aus ordentlichen Mitgliedern, Fördermitgliedern und
Ehrenmitgliedern.

'S Ordentliches Mitglied kann jede natürliche Person werden, die das 14.
Lebensjahr vollendet hat.

'S Fördermitglieder können sowohl natürliche unbeschränkt geschäftsfähige
Personen wie auch juristische Personen werden.
'S Auf sie finden die Regelungen für ordentliche Mitglieder entsprechende
Anwendung, soweit diese Satzung nichts Abweichendes bestimmt.

'S Ehrenmitglieder werden von der Mitgliederversammlung gewählt und ernannt.
'S Sie müssen nicht bereits Mitglieder des Vereins sein.
'S Die Ehrenmitgliedschaft kann jederzeit durch die Mitgliederversammlung
wieder entzogen werden.
'S Auf Ehrenmitglieder finden die Regelungen für ordentliche Mitglieder
entsprechende Anwendung, soweit diese Satzung nichts Abweichendes bestimmt.

\Clause{title={Beginn und Ende der Mitgliedschaft}}

'S Die Mitgliedschaft kann gegenüber dem Vorstand in Textform beantragt werden.
'S Bei beschränkt geschäftsfähigen Minderjährigen ist zur Aufnahme in den
Verein die Zustimmung des gesetzlichen Vertreters in Schriftform erforderlich.
'S Über die Annahme des Antrags entscheidet der Vorstand.
'S Gegen die Ablehnung, die keiner Begründung bedarf, steht dem Bewerber die
Berufung an die nächste ordentliche Mitgliederversammlung zu.
'S Diese entscheidet endgültig.

'S Die Mitgliedschaft endet durch freiwilligen Austritt, Streichung von der
Mitgliederliste, Ausschluss, Tod des Mitglieds oder mit dauerhaftem Verlust der
Rechtsfähigkeit.

'S Die Kündigung der ordentlichen und der Fördermitgliedschaft kann dem
Vorstand gegenüber unter Einhaltung einer Frist von einem Monat zum Ende des
Quartals in Textform erklärt werden.
'S Die Rückgabe der Ehrenmitgliedschaft ist jederzeit ohne Einhaltung von
Frist- und Formerfordernissen möglich.

'S Ein ordentliches Mitglied kann durch Vorstandsbeschluss von der
Mitgliederliste gestrichen werden, wenn es einen bestehenden Beitragsrückstand
auch zwei Wochen nach Verschicken der zweiten Mahnung nicht vollständig
ausgeglichen hat.
'S Das Nähere zum Mahnwesen regelt die Beitragsordnung.
'S Die Streichung ist dem Mitglied in Textform mitzuteilen.
'S Sie ist keinem Rechtsbehelf zugänglich.

'S Wenn ein Mitglied das Ansehen des Vereins in der Öffentlichkeit schädigt,
wiederholt seinen Beitragsverpflichtungen nur nach Mahnung nachkommt,
wiederholt der Haus- und Benutzungsordnung zuwiderhandelt, grob gegen die
Vereinsinteressen verstoßen hat oder ein sonstiger Grund vorliegt, kann es vom
Verein ausgeschlossen werden.
'S Über den Ausschluss von ordentlichen Mitgliedern entscheidet der Vorstand.
'S Über den Ausschluss von Förder- und Ehrenmitgliedern entscheidet die
Mitgliederversammlung.
'S Dem auszuschließenden Mitglied ist der Beschluss in Textform unter Angabe
einer Begründung mitzuteilen.
'S Soweit der Ausschluss auf einer Entscheidung des Vorstandes beruht, ist die
Berufung an die nächsten ordentlichen Mitgliederversammlung zulässig.
'S Bis zur endgültigen Entscheidung der Mitgliederversammlung über den
Ausschluss, bei der die Stimme des betroffenen Mitglieds unberücksichtigt
bleibt, ruht die Mitgliedschaft unter Aussetzung der Beitragspflicht.

'S Mit Beendigung der Mitgliedschaft erlöschen alle Ansprüche gegen den Verein
aus der Mitgliedschaft.
'S Eine Rückgewähr von Beiträgen, Spenden oder anderen Zuwendungen und
Unterstützungsleistungen erfolgt nicht.

\Clause{title={Rechte und Pflichten der Mitglieder}}

'S Ordentliche Mitglieder besitzen das aktive und passive Wahlrecht sowie das
Antrags-, Stimm- und Rederecht in der Mitgliederversammlung.
'S Jedes ordentliche Mitglied hat gleiches Stimm- und Wahlrecht in der
Mitgliederversammlung.
'S Ein ordentliches Mitglied kann kraft Vollmacht maximal vier Stimmrechte von
anderen ordentlichen Mitgliedern ausüben.
'S Die Bevollmächtigung ist gegenüber dem Versammlungsleiter zum Zeitpunkt des
Zusammentretens der Mitgliederversammlung nachzuweisen.

'S Förder- und Ehrenmitglieder besitzen in der Mitgliederversammlung lediglich
ein Rederecht.

'S Alle Mitglieder unterstützen den Verein – auch in der Öffentlichkeit.
'S Sie haben Mitgliedsbeiträge und die Aufnahmegebühr zu entrichten, soweit
diese Satzung oder die Beitragsordnung dies festlegen.
'S Ordentliche Mitglieder haben darüber hinaus Pflichtdienste zu leisten,
soweit diese Satzung oder die Beitragsordnung dies festlegen.
'S Ferner sind von allen Mitgliedern die Haus-, Benutzungs- und
Geschäftsordnungen des Vereins einzuhalten.

'S Kosten, die den Mitgliedern durch ihre Vereinsarbeit notwendiger Weise
entstanden sind, können vom Verein ersetzt werden, soweit diese Kosten nach dem
Grundsatz der Sparsamkeit und Wirtschaftlichkeit angemessen sind.
'S Über den Ersatz solcher Aufwendungen entscheidet der Vorstand.

'S Der Verein kann für durch seine Mitglieder erbrachte Vereinsarbeit, die das
normale ehrenamtliche Engagement übersteigt, eine angemessene
Aufwandsentschädigung nach § 3 Nr. 26a EStG zahlen, sofern dies im Hinblick auf
den tatsächlich entstandenen Aufwand sowie die finanziellen Mittel des Vereins
verhältnismäßig ist und der Förderung des Vereinszwecks dient.
'S Werden Mitglieder über die übliche Vereinsarbeit hinaus für den Verein
tätig, kann der Verein eine Aufwandsentschädigung gemäß § 3 Nr. 26 EStG zahlen
oder die Tätigkeit auf Grundlage eines Dienst-, Honorar- oder Werkvertrages
vergüten, sofern dies im Hinblick auf den tatsächlich entstandenen Aufwand
sowie die finanziellen Mittel des Vereins verhältnismäßig ist und der Förderung
des Vereinszwecks dient.
'S Über die Zahlung von Aufwandsentschädigungen, die Vertragsbedingungen und
Vertragsinhalte sowie für eine eventuelle Vertragsbeendigung entscheidet der
Vorstand.

\Clause{title={Aufnahmegebühr und Beiträge}}

'S Für die Erfüllung der satzungsmäßigen Zwecke werden Mittel verwendet, die
insbesondere durch Beiträge, Spenden und Zuschüsse erlangt werden.
'S Die Beitragsordnung regelt die Höhe der Mitgliedsbeiträge und der
Aufnahmegebühr sowie die organisatorischen Abläufe ihrer Erhebung.
'S Sie wird von der Mitgliederversammlung mit einer Zweidrittelmehrheit der
abgegebenen Stimmen beschlossen.

'S Ehrenmitglieder sind von der Aufnahmegebühr und der Beitragspflicht befreit.

'S Die Mitgliederversammlung kann einzelne Mitglieder nach eigenem Ermessen
ganz oder teilweise von der Aufnahmegebühr und der Beitragspflicht befreien.
'S Die Beitragsordnung kann auch für den Vorstand Kompetenzen zur Befreiung von
Aufnahmegebühr und Beitragspflicht vorsehen.

\Clause{title={Organe des Vereins}}

'S Organe des Vereins sind
\begin{enumerate}
	\item die Mitgliederversammlung,
	\item der Vorstand und
	\item der Beirat.
\end{enumerate}

\Clause{title={Die Mitgliederversammlung}}

'S Oberstes Organ des Vereins ist die Mitgliederversammlung.
'S Sie hat über grundsätzliche Fragen und Angelegenheiten des Vereins zu
beschließen.
'S Dabei hat sie insbesondere folgende Aufgaben:
\begin{enumerate}
	\item Wahl des Vorstands und der Kassenprüfer,
	\item Beschlussfassung über die Satzung, über Änderungen der Satzung,
	      einschließlich der Änderung des Vereinszwecks, und die über
		  Auflösung des Vereins,
	\item Entscheidung über die Entlastung des Vorstands,
	\item Beschluss über vorliegende Anträge, insbesondere auch über die
	      Aufnahme und den Ausschluss von Mitgliedern, soweit dies die Satzung
		  vorsieht,
	\item Zustimmung zu Grundstücksgeschäften und zur Aufnahme von Darlehen,
	\item Entscheidungen über einen pauschalierten Ersatz der Aufwendungen des
	      Vorstands und über die Zahlung einer Pauschale gemäß § 3 Nr. 26a EStG
		  an den Vorstand,
	\item Genehmigung sämtlicher Haus-, Benutzungs- und Geschäftsordnungen des
	      Vereins,
	\item Beschlussfassung über die Beitragsordnung.
\end{enumerate}

'S Die ordentliche Mitgliederversammlung findet mindestens einmal im Jahr
statt, möglichst im ersten Quartal.
'S Der Vorstand hat daneben außerordentliche Mitgliederversammlungen
einzuberufen, wenn es das Interesse des Vereins erfordert oder wenn ein Viertel
der Mitglieder dies schriftlich unter Angabe des Zwecks und nebst einer
Begründung beantragt.

'S Die Mitgliederversammlung kann über geeignete Online-Kommunikation
stattfinden, wenn sichergestellt ist, dass alle stimmberechtigten Mitglieder
zumutbar teilnehmen können.

'S Die Mitgliederversammlung wird durch den Vorstand einberufen, er setzt
rechtzeitig durch Beschluss einen Termin fest.
'S Die Einladung zur Mitgliederversammlung erfolgt per E-Mail; ihr ist eine vom
Vorstand festgelegte Tagesordnung beizufügen.
'S Über Anträge zur Tagesordnung, die vom Vorstand nicht aufgenommen wurden
oder die erstmals in der Mitgliederversammlung gestellt werden, entscheidet die
Mitgliederversammlung.
'S Satz 3 gilt nicht für solche Anträge, die eine Änderung der Satzung,
einschließlich der Änderung des Vereinszwecks, die Auflösung des Vereins oder
Änderungen der Beitragsordnung zum Gegenstand haben.
'S Die Ladung zur Mitgliederversammlung hat unter Einhaltung einer Frist von 21
Tagen zu erfolgen.
'S Ist in besonderen Fällen ein zeitnaher Beschluss der Mitgliederversammlung
erforderlich, kann die Ladung zu einer außerordentlichen Mitgliederversammlung
auch unter Einhaltung einer verkürzten angemessenen Frist erfolgen.
'S Das Vorliegen eines besonderen Falles ist in der Einladung zu begründen.

'S Die Mitgliederversammlung wird vom Ersten Vorsitzenden, bei dessen
Verhinderung vom Zweiten Vorsitzenden oder hilfsweise dem Schatzmeister
geleitet.
'S Der Versammlungsleiter benennt einen Protokollführer.

'S Hat der Verein bis zu 45 ordentliche Mitglieder, so ist die
Mitgliederversammlung beschlussfähig, wenn mindestens ein Drittel der Stimmen
aller ordentlichen Mitglieder repräsentiert wird.
'S Hat der Verein demgegenüber mehr als 45 ordentliche Mitglieder, so ist die
Mitgliederversammlung beschlussfähig, wenn die Stimmrechte von mindestens 15
ordentlichen Mitgliedern repräsentiert werden.
'S Bei Beschlussunfähigkeit ist der Vorstand verpflichtet, innerhalb von
maximal 21 Tagen eine zweite Mitgliederversammlung mit identischer Tagesordnung
einzuberufen.
'S Diese ist bei Teilnahme von drei ordentlichen Mitgliedern beschlussfähig.
'S Hierauf ist in der Einladung hinzuweisen.
'S Wenn die Mitgliederversammlung nicht gemäß Absatz 3 unter Zuhilfenahme von
geeigneten Online-Kommunikationsmitteln stattfindet, hat sie innerhalb der
Stadt Oldenburg zusammenzutreten.

'S Die Mitgliederversammlung fasst ihre Beschlüsse mit einfacher Mehrheit der
abgegebenen Stimmen (einfache Abstimmungsmehrheit), soweit diese Satzung nichts
Abweichendes bestimmt.
'S Das bedeutet, dass diejenige Beschlussvorlage angenommen wird, die mehr als
die Hälfte der abgegebenen Stimmen auf sich vereinigen kann.
'S Stimmenthaltungen und ungültige Stimmen bleiben bei der Ermittlung
sämtlicher Abstimmungsmehrheiten stets unberücksichtigt.
'S Bei Stimmgleichheit gilt eine Beschlussvorlage als abgelehnt.
'S Bei Personenwahlen ist, soweit diese Satzung nichts Abweichendes bestimmt,
ebenfalls die einfache Mehrheit der abgegebenen Stimmen (einfache
Abstimmungsmehrheit) zum Obsiegen notwendig.
'S Hat im ersten Wahlgang kein Kandidat die einfache Abstimmungsmehrheit
erreicht, findet eine Stichwahl zwischen denjenigen Kandidaten statt, welche
die höchsten Stimmenzahlen auf sich vereinen konnten.
'S Sieger der Stichwahl ist derjenige Kandidat, der die meisten abgegebenen
Stimmen auf sich vereinigen kann (relative Abstimmungsmehrheit).
'S Stehen im ersten Wahlgang weniger als drei Kandidaten zur Wahl, ist
abweichend von Satz 5 der Kandidat gewählt, der die meisten abgegebenen Stimmen
auf sich vereinigen kann (relative Abstimmungsmehrheit).
'S Abstimmungen und Wahlen finden grundsätzlich nicht geheim statt.
'S Ein Antrag auf geheime Abstimmung oder Wahl ist nur bei
Mitgliederversammlungen zulässig, die nicht unter Zuhilfenahme von geeigneten
Online-Kommunikationsmitteln gemäß Absatz 3 stattfinden.
'S Über die Annahme eines solchen Antrags entscheidet die Mitgliederversammlung
in nichtgeheimer Abstimmung.

'S Die Mitgliederversammlung tagt nicht öffentlich.
'S Über die Zulassung von Gästen und Medienvertretern entscheidet die
Mitgliederversammlung zu Beginn ihres Zusammentretens.

'S Die Ergebnisse der Mitgliederversammlung werden protokolliert und zumindest
allen ordentlichen Mitgliedern binnen Monatsfrist mitgeteilt.
'S Die Protokolle werden vom Ersten und vom Zweiten Vorsitzenden sowie vom
Protokollführer unterzeichnet.

\Clause{title={Der Vorstand}}

'S Der Vorstand besteht aus
\begin{enumerate}
	\item dem Ersten Vorsitzenden,
	\item dem Zweiten Vorsitzenden und
	\item dem Schatzmeister.
\end{enumerate}
'S Eine Personalunion zwischen mehreren Vorstandsämtern ist unzulässig.

'S Die Vorstandsmitglieder werden von der Mitgliederversammlung einzeln für
zwei Jahre gewählt.
'S Wählbar ist jedes unbeschränkt geschäftsfähige ordentliche Mitglied.
'S Die Wiederwahl ist zulässig.
'S Scheidet ein Mitglied des Vorstandes während der Amtsperiode aus, so ist
zeitnah eine Mitgliederversammlung zur Nachwahl eines Ersatzmitglieds für die
restliche Amtsdauer des ausgeschiedenen Vorstandsmitglieds einzuberufen.
'S Nach Ablauf der Amtszeit bleibt der Vorstand bis zur Konstituierung des
neuen Vorstandes im Amt.

'S Die drei Vorstandsmitglieder vertreten den Verein gerichtlich und
außergerichtlich vorbehaltlich der Beschränkungen in den Sätzen 2 bis 4 in
Einzelvertretung.
'S Bei Rechtsgeschäften, deren Geschäftswert 1.500,00 Euro überschreitet, ist
die gemeinschaftliche Vertretung durch zwei Vorstandsmitglieder erforderlich.
'S Beim Erwerb oder Verkauf von Grundstücken, bei der Belastung und bei allen
sonstigen Verfügungen über Grundstücke oder grundstücksgleiche Rechte ist die
Vertretungsmacht des Vorstands mit Wirkung gegen Dritte in der Weise
beschränkt, dass die Zustimmung der Mitgliederversammlung hierzu erforderlich
ist.
'S Die Beschränkung aus Satz 3 gilt auch für die Aufnahme von Darlehen in
jeglicher Höhe.

'S Die drei Vorstandsmitglieder verfügen jeweils einzeln über die Bankkonten
des Vereins.

'S Der Vorstand führt die Geschäfte des Vereins.
'S Er tritt nach Bedarf zusammen und fasst seine Beschlüsse im Allgemeinen in
Vorstandssitzungen, die vom Ersten Vorsitzenden schriftlich, fernmündlich,
mündlich oder per E-Mail unter Einhaltung einer angemessenen Frist einberufen
werden.
'S Der Vorstand ist beschlussfähig, wenn mindestens zwei Vorstandsmitglieder
anwesend sind.
'S Der Vorstand fasst seine Beschlüsse mit einfacher Abstimmungsmehrheit.
'S Bei Stimmengleichheit entscheidet die Stimme des Ersten Vorsitzenden, bei
dessen Verhinderung die des Zweiten Vorsitzenden.
'S Ein Vorstandsbeschluss kann auch auf schriftlichem Wege, fernmündlich oder
unter Zuhilfenahme von geeigneten Online-Kommunikationsmitteln gefasst werden
soweit alle Vorstandsmitglieder ihre Zustimmung zu der Beschlussvorlage
erklären.
'S In unvorhergesehenen unaufschiebbaren Fällen können einzelne Mitglieder des
Vorstands selbstständig Entscheidungen ohne Beratung und Beschlussfassung
treffen (Eilkompetenz).
'S Über solche Entscheidungen ist der übrige Vorstand umgehend zu informieren.

'S Der Vorstand ist den Grundsätzen einer ordnungsgemäßen Buchführung sowie dem
Prinzip der effektiven und sparsamen Mittelverwendung verpflichtet.
'S Zur Mittelverwendung bedarf es keines von der Mitgliederversammlung
beschlossenen Haushaltsplans.
'S Allerdings bedarf es zur Mittelverwendung, unabhängig von deren Höhe, stets
eines Vorstandsbeschlusses.
'S Ausgaben dürfen nur auf Guthabenbasis getätigt werden.
'S Freie Rücklage bleiben bei der Ermittlung des Guthabens stets
unberücksichtigt.
'S Zweckgebundene Rücklagen bleiben bei der Ermittlung des Guthabens ebenfalls
unberücksichtigt, soweit sie nicht ausdrücklich zum Zwecke der entsprechenden
Mittelverwendung gebildet wurden.
'S Über Abweichungen von den Sätzen 4 bis 6 entscheidet die
Mitgliederversammlung im Einzelfall.

'S Der Schatzmeister führt die Finanzgeschäfte des Vereins und ist für die
Buchführung verantwortlich.
'S Im Rahmen dessen ist er insbesondere zuständig für die Kontrolle und
Steuerung der Liquidität des Vereins, das Beitragswesen, sämtliche
steuerrechtliche Angelegenheiten und die Erstellung des Jahresabschlusses.
'S Der Jahresabschluss ist in Schriftform im ersten Quartal des auf den
Berichtszeitraum folgenden Jahres, spätestens zur ordentlichen
Mitgliederversammlung, fertigzustellen.

'S Vorstandsbeschlüsse und Entscheidungen kraft Eilkompetenz sind zu
protokollieren und für die ordentlichen Vereinsmitgliedern zum Abruf
elektronisch zu hinterlegen.
'S Der Vorstand legt einen jährlichen Tätigkeitsbericht in Schriftform im
ersten Quartals des auf den Berichtszeitraum folgenden Jahres, spätestens zur
ordentlichen Mitgliederversammlung, vor.

'S Der Vorstand führt die Geschäfte des Vereins ehrenamtlich.
'S Er hat Anspruch darauf, dass ihm seine im Zuge der Vorstandsarbeit
entstandenen Aufwendungen ersetzt werden.
'S Der Ersatz kann auch pauschaliert erfolgen, solange die Pauschale die
tatsächlich entstandenen Kosten nicht übersteigt.
'S Daneben ist die Zahlung einer Pauschale gemäß § 3 Nr. 26a EStG zulässig.
'S Über die Zahlung von Pauschalen nach Satz 3 und Satz 4 entscheidet dem
Grunde und der Höhe nach die Mitgliederversammlung.
'S Vorstandsmitglieder können auch über ihre Vorstandstätigkeit hinaus für den
Verein tätig werden und dafür eine Aufwandsentschädigung gemäß § 3 Nr. 26 EStG
erhalten oder solche Tätigkeiten entgeltlich auf Grundlage eines Dienst-,
Honorar- oder Werkvertrages ausüben, sofern dies im Hinblick auf den tatsächlich
entstandenen Aufwand, sowie die finanziellen Mittel des Vereins verhältnismäßig
ist und der Förderung des Vereinszwecks dient.
'S Über die Zahlung von Aufwandsentschädigungen nach Satz 6, sowie die
diesbezüglichen Vertragsbedingungen und Vertragsinhalte, sowie für eine
eventuelle Vertragsbeendigung entscheidet der Vorstand ohne die Stimme des
jeweils betroffenen Vorstandsmitglieds.

\Clause{title={Der Beirat}}

'S Durch Beschluss der Mitgliederversammlung kann ein Beirat eingerichtet
werden.
'S Dieser berät den Verein und unterstützt die Erreichung des Vereinszwecks
vornehmlich ideell.
'S Die Beiratsmitglieder werden von der Mitgliederversammlung mit einfacher
Abstimmungsmehrheit einzeln berufen und abberufen.
'S Die Mitglieder des Beirats müssen nicht Mitglieder des Vereins sein.

'S Der Beirat organisiert sich selbst.
'S Dazu kann er sich eine Geschäftsordnung geben, diese Bedarf nicht der
Zustimmung der Mitgliederversammlung.

\Clause{title={Kassenprüfung}}

'S Die Mitgliederversammlung wählt zwei Kassenprüfer für die Dauer von einem
Jahr.
'S Die Wiederwahl ist zulässig.
'S Die Kassenprüfer dürfen dem Vorstand nicht angehören und nicht in einem
Arbeitsverhältnis mit dem Verein stehen.

'S Bei Ausscheiden eines Kassenprüfers ernennt dieser einen Nachfolger,
hilfsweise bestimmt ihn die Mitgliederversammlung nach Maßgabe von Absatz 1.

'S Die Kassenprüfer haben die Aufgabe, Rechnungsbelege sowie deren
ordnungsgemäße Verbuchung und die Mittelverwendung gemeinschaftlich zu prüfen,
insbesondere die satzungsgemäße korrekte Mittelverwendung unter Einhaltung der
Grundsätze der ordnungsgemäßen Buchhaltung.
'S Die Prüfung erstreckt sich nicht auf die Zweckmäßigkeit der vom Vorstand
getätigten Ausgaben.
'S Der Schatzmeister hat die Kassenprüfer bei ihrer Tätigkeit zu unterstützen.
'S Die Kassenprüfer unterrichten die Mitgliederversammlung über das Ergebnis
der Prüfung durch einen gemeinsamen Bericht in Schriftform.

\Clause{title={Satzungsänderungen}}

'S Die Mitgliederversammlung beschließt Änderungen dieser Satzung mit einer
Zweidrittelmehrheit der abgegebenen Stimmen.
'S Absatz 1 findet ebenfalls bei Änderung des Vereinszwecks Anwendung.

\Clause{title={Auflösung des Vereins}}

'S Die Auflösung des Vereins kann nur mit einer Zweidrittelmehrheit der
abgegebenen Stimmen der Mitgliederversammlung beschlossen werden.

'S Bei Auflösung des Vereins oder bei Wegfall seiner bisherigen gemeinnützigen
Zwecke ist das gesamte Vereinsvermögen einer gemeinnützigen und
steuerbegünstigten, besonders anerkannten Institution zuzuführen, die ebenfalls
dem Wesen des Vereinszwecks entspricht.
'S Die Institution wird von der Mitgliederversammlung gemeinsam mit der
Auflösung des Vereins beschlossen.
'S Für die Bestimmung der Institution genügt die relative Abstimmungsmehrheit.

'S Die amtierenden vertretungsberechtigten Vorstandsmitglieder werden zu
Liquidatoren ernannt, soweit die Mitgliederversammlung keinen abweichenden
Beschluss fasst.

\Clause{title={Form- und Schlussbestimmungen}}

'S Soweit diese Satzung das Erfordernis der Textform vorsieht, kann dieses
insbesondere durch Verwendung einer einfachen E-Mail eingehalten werden.
'S Sieht diese Satzung demgegenüber die Schriftform vor, so gilt das strenge
Schriftformerfordernis gemäß § 126 BGB (eigenhändige Unterschrift).

'S Sollte eine Bestimmung dieser Satzung unwirksam oder undurchführbar sein,
wird die Wirksamkeit der übrigen Bestimmungen davon nicht berührt.
'S Die Mitglieder verpflichten sich, anstelle einer unwirksamen oder
undurchführbaren Bestimmung eine dieser Bestimmung möglichst nahekommende
wirksame Regelung zu treffen.

'S Soweit in dieser Satzung nichts Abweichendes geregelt ist, gelten die
gesetzlichen Vorschriften.
\end{contract}

\end{document}
