\documentclass[a4paper,10pt]{scrreprt}
\usepackage[utf8]{inputenc}
\usepackage[ngerman]{babel}
\usepackage{lmodern}
\usepackage[juratotoc]{scrjura}
\usepackage[]{scrpage2}
\usepackage{graphicx}
\usepackage[left=3cm, right=2.5cm, top=2.65cm, bottom=2.65cm]{geometry}

\newcommand{\shiftleft}[2]{\makebox[0pt][r]{\makebox[#1][l]{#2}}}

\rofoot {
	\raisebox{-2.7cm}{
		\shiftleft{0.88\paperwidth}{
			\includegraphics[width=\paperwidth]{img/footer-line.pdf}
		}
	}
}

\thispagestyle{scrheadings}
\pagestyle{scrheadings}

\sloppy

\useshorthands{'}
\defineshorthand{'S}{\Sentence\ignorespaces}

\setlength{\parindent}{0pt}
\setlength{\parskip}{10pt}

\usepackage{hyperref}
\hypersetup{
	pdfauthor={Kreativität trifft Technik e.V.},
	pdftitle={Beitragsordnung des Vereins Kreativität trifft Technik}
}

\begin{document}

\title{Beitragsordnung des Vereins\\Kreativität trifft Technik}
\subtitle{Beschlossen auf der Gründungsversammlung am 11. Juli 2011}
\author{Kreativität trifft Technik}

\makeatletter
\begin{titlepage}

\newcommand{\HRule}{\rule{\linewidth}{0.5mm}} % Defines a new command for the horizontal lines, change thickness here

\center % Center everything on the page
 
%----------------------------------------------------------------------------------------
%   HEADING SECTIONS
%----------------------------------------------------------------------------------------

\textsc{\LARGE \@author}\\[1.5cm] % Name of your university/college

%----------------------------------------------------------------------------------------
%   TITLE SECTION
%----------------------------------------------------------------------------------------

\HRule \\[0.4cm]
{ \huge \bfseries \@title}\\[0.4cm] % Title of your document
\HRule \\[1.5cm]
 
% If you don't want a supervisor, uncomment the two lines below and remove the section above
%\Large \emph{Author:}\\
%John \textsc{Smith}\\[3cm] % Your name

%----------------------------------------------------------------------------------------
%   DATE SECTION
%----------------------------------------------------------------------------------------

{\large \@subtitle}\\[2cm] % Date, change the \today to a set date if you want to be precise

%----------------------------------------------------------------------------------------
%   LOGO SECTION
%----------------------------------------------------------------------------------------

\includegraphics[scale=2.0]{img/logo.pdf}\\[1cm] % Include a department/university logo - this will require the graphicx package
 
%----------------------------------------------------------------------------------------

\vfill % Fill the rest of the page with whitespace

\end{titlepage}


\begin{contract}

\Paragraph{title={Mitgliedsbeiträge}}

'S Der monatliche Mitgliedsbeitrag für ordentliche Mitglieder beträgt 25,00
Euro.
'S Sofern ordentliche Mitglieder Angehörige einer in § 3 Absatz 1 dieser
Beitragsordnung genannten Personengruppe sind, können sie abweichend von
Satz 1 einen ermäßigten monatlichen Mitgliedsbeitrag in Höhe von 10,00
Euro entrichten.

'S Der monatliche Mitgliedsbeitrag für Fördermitglieder beträgt 5,00 Euro.

'S Ehrenmitglieder sind von der Beitragspflicht befreit.

'S Nach Maßgabe von § 3 Absatz 4 dieser Beitragsordnung kann der Vorstand
ordentliche Mitglieder von der Beitragspflicht ganz oder teilweise befreien
sowie Stundungsabreden treffen.

\Paragraph{title={Aufnahmegebühren}}

'S Die Aufnahmegebühr für ordentliche Mitglieder und Fördermitglieder beträgt
10,00 Euro.

'S Ehrenmitglieder sind von der Aufnahmegebühr befreit.

'S Nach Maßgabe von § 3 Absatz 4 dieser Beitragsordnung kann der Vorstand
ordentliche Mitglieder von der Zahlung der Aufnahmegebühr ganz oder teilweise
befreien sowie Stundungsabreden treffen.

\Paragraph{title={Ermäßigungen und Befreiungen von der Zahlungspflicht}}

'S Angehörigen der nachfolgend genannten Personengruppen steht eine Ermäßigung
des Mitgliedsbeitrags gemäß § 1 Absatz 1 Satz 2 dieser Beitragsordnung zu:
\begin{enumerate}
	\item Schüler, Studenten, Referendare und Auszubildende.
	\item Rentner.
	\item Personen, die Wehrdienst, den Bundesfreiwilligendienst, ein
          Freiwilliges Soziales Jahr oder ein Freiwilliges Ökologisches
          Jahr ableisten.
	\item Arbeitslose.
	\item Empfänger von Transferleistungen nach SGB II und SGB XII.
\end{enumerate}

'S Auf Nachfrage ist dem Vorstand über die Zugehörigkeit zu einer der in Absatz
1 genannten Personengruppe ein entsprechender Nachweis vorzulegen.

'S Gemäß § 6 Absatz 3 Satz 1 der Satzung kann die Mitgliederversammlung mit
einfacher Mehrheit Mitglieder von der Beitragspflicht ganz oder teilweise
befreien.
'S Dies ist auch rückwirkend möglich.

'S Der Vorstand kann unter Bezugnahme auf § 6 Absatz 3 Satz 2 der Satzung
selbstständig und nach eigenem Ermessen über eine angemessene oder sogar
vollständige Beitragsbefreiung oder eine Beitragsstundung abschließend
entscheiden.
'S Gleiches gilt für die Befreiungen von der Pflicht zur Zahlung der
Aufnahmegebühr.
'S Eine Veröffentlichung der Vorstandsbeschlüsse über solche
Zahlungsbefreiungen und Stundungen erfolgt anonymisiert im Jahresbericht.
'S Intern sind namentliche Aufzeichnungen vorzuhalten.
'S Über die Einsichtnahme in diese Aufzeichnungen entscheidet unter Beachtung
der einschlägigen datenschutzrechtlichen Regelungen die Mitgliederversammlung
mit einfacher Abstimmungsmehrheit.

'S Eine rückwirkende Rücknahme einer Befreiung von der Beitragspflicht ist
außer in Fällen ungerechtfertigter Bereicherung unzulässig.

\Paragraph{title={Fälligkeit und Zahlungsweise}}

'S Die Mitgliedsbeiträge werden jeweils zum Quartalsanfang (d.h. zum 01.
Januar, 01. April, 01. Juli und 01. Oktober) im Voraus fällig.
'S Die Zahlung ist per Lastschriftverfahren mittels Einzugsermächtigung zu
leisten.
'S Im Falle von Rücklastschriften verpflichtet sich das Mitglied, alle dem
Verein durch die Rücklastschrift entstandenen Aufwendungen und Auslagen zu
ersetzen.
'S Bei Beitritt zum Verein innerhalb eines laufenden Quartals ist der
Mitgliedsbeitrag für das Quartal anteilig, inklusive des Monats, in den der
Beitritt fällt, sofort fällig.

'S Die Aufnahmegebühr wird mit Annahme des Aufnahmeersuchens in voller Höhe
fällig und ist per Lastschriftverfahren mittels Einzugsermächtigung zu
entrichten.
'S Absatz 1 Satz 3 gilt auch für die Zahlung der Aufnahmegebühr.

'S In Ausnahmefällen kann auch eine Barzahlung an den Schatzmeister geleistet
werden, sofern dieser zum entsprechenden Zeitpunkt zur Entgegennahme bereit
ist.

\Paragraph{title={Mahnwesen und Inkasso}}

'S Mitglieder, die mit der Zahlung ihres Beitrages mehr als einen Monat in
Rückstand sind, sind in Schriftform zu mahnen.
'S Bleibt die Mahnung erfolglos, ist sie nach einem weiteren Monat zu
wiederholen.

'S Über Inkassomaßnahmen jeder Art entscheidet der Schatzmeister.

\Paragraph{title={Pflichtdienste}}

'S Pflichtdienste sind nicht vorgesehen.

\end{contract}
\end{document}
